\documentclass[]{article}
\usepackage{caption,subcaption,graphicx,float,url,amsmath,amssymb,amsthm,tocloft,cancel,thmtools}
\newtheorem{ex}{Exercise}
\newcommand\numberthis{\addtocounter{equation}{1}\tag{\theequation}}
%opening
\title{Exercises from Strogatz, Chapter 8}
\author{Simon Crase}

\begin{document}

\maketitle

\begin{abstract}
This document consists of worked exampled from \cite[Chapter 8]{strogatz:2000}.
\end{abstract}

\begin{ex}[Exercise 8.7.1]
	\begin{enumerate}
		\item Use partial fractions to evaluate the integral $\int_{r_0}^{r_1}\frac{dr}{r(1-r^2)}$ that arise in \cite[Example 8.7.1]{strogatz:2000}, and show that $r_1=P(r)\triangleq\frac{1}{\sqrt{1+e^{-4\pi}(\frac{1}{r_0^2}-1)}}$.\label{item:8.7.1.1}
		\item  Then confirm that $P'(r^*)=e^{-4\pi}$.\label{8.7.1.2}
	\end{enumerate}
\end{ex}

\begin{proof}
	\begin{align*}
	\frac{1}{r(1-r^2)}=&\frac{A}{r} + \frac{B}{1-r} + \frac{C}{1+R}\text{, whence}\\
	1 =& A(1-r^2) + B r(1+r) + Cr(1-r)\\
	=& A -A r^2 + B + B r^2 +Cr -Cr^2\text{. Equating coeficient of $r$}\\
	A + C =& B\\
	B + C =& 0\\
	A=& 1,\text{, whence $B=\frac{1}{2}$, and $C=\frac{-1}{2}$, so}\\
	\int_{r_0}^{r_1}\frac{dr}{r(1-r^2)}=&\int_{r_0}^{r1} \big[\frac{1}{r}-\frac{1}{2(r-1)}-\frac{1}{2(1+r)}\big] dr\\
	=&\bigg[\ln(r)-\frac{1}{2}\ln(r-1)-\frac{1}{2}\ln(r+1)\bigg]_{r_0}^{r_1}\\
	=&\bigg[\ln(\frac{r}{\sqrt{r^2-1}})\bigg]_{r_0}^{r_1}\\
	=2&\pi\text{, \cite[Example 8.7.1]{strogatz:2000}so that}\\
	\ln(\frac{r_1}{\sqrt{r_1^2-1}}\frac{\sqrt{r_0^2-1}}{r_0})=&2 \pi\\
	\frac{r_1}{\sqrt{r_1^2-1}}=&\frac{r_0}{\sqrt{r_0^2-1}} e^{2 \pi}\\
	\frac{r_1^2}{r_1^2-1}=&\frac{r_0^2}{r_0^2-1} e^{4 \pi}\\
	=&K^2{, say}\numberthis\label{eq:exercise_8_6_1_K}
	\end{align*}
	\begin{align*}
	\frac{r_1^2}{r_1^2-1}=&K^2\\
	r_1^2=&K^2 r_1^2 - K^2\\
	K^2 =& (K^2-1)r_1^2\\
	r_1=&\frac{K}{\sqrt{K^2-1}}\\
	=&\frac{1}{\sqrt{1-\frac{1}{K^2}}}\text{, from (\ref{eq:exercise_8_6_1_K})}\\
	=&\frac{1}{\sqrt{1-\frac{1}{e^{4 \pi} \frac{r_0^2}{r_0^2-1}}}}\\
	=&\frac{1}{\sqrt{1-e^{-4\pi}(1-\frac{1}{r_0^2})}}\\
	=&\frac{1}{\sqrt{1+e^{-4\pi}(\frac{1}{r_0^2}-1)}}
	\end{align*}
	Which proves Item \ref{item:8.7.1.1}. Moreover
	\begin{align*}
	P'(r) =& \frac{d}{dr} \frac{1}{\sqrt{1+e^{-4\pi}(\frac{1}{r^2}-1)}}\\
	=& \frac{d}{dr} \bigg[1+e^{-4\pi}(r^{-2}-1)\bigg]^{-\frac{1}{2}}\\
	=&  \frac{-1}{2}\bigg[1+e^{-4\pi}(r^{-2}-1)\bigg]^{-\frac{3}{2}} e^{-4\pi} (-2 r^{-3})\\
	=&  \bigg[1+e^{-4\pi}(r^{-2}-1)\bigg]^{-\frac{3}{2}} e^{-4\pi}  r^{-3}\text{, so}\\
	P'(1) =&  e^{-4\pi}
	\end{align*}
	Which proves Item \ref{8.7.1.2}.
\end{proof}



\bibliographystyle{unsrt}
\addcontentsline{toc}{section}{Bibliography}
\bibliography{../dynamics}

\end{document}
